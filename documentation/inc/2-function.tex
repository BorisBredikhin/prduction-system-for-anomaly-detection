\section{Описание работы системы}

\subsection{Со стороны конечного пользователя}

При запуске системы в память загружается база знаний, в которой указаны правила и список начальных переменных.
Затем пользователю предлагается вести значения этих переменных.

После ввода значений система делает несколько проходов по базе правил, возможно, создавая при этом вспомогательные переменные и давая экспертные рекомендации.
При достижении правила с меткой <<\texttt{result}>> система делает вывод и завершает свою работу.

\subsection{Со стороны эксперта}

Эксперт может:
\begin{itemize}
	\item Просмотреть правила
	\item Добавить    правила
	\item Изменить    правила
	\item Удалить     правила
	\item \textit{Проверить базу правил на совместность} % Возможно
\end{itemize}
