\section{Архитектура системы}

Предполагается следующая архитектура системы (рис. \ref{fig:ais}).

\addimghere{img/system_architecture}{1.0}{Архитектура интеллектуальной системы}{fig:ais}

Для построения интерфейсов пользователя будет использоваться tkinter, база данных будет представлена в виде словаря Python, а база знаний будет храниться в JSON.

Работа системы основана на очереди сообщений.

\subsection{База знаний}

Правила задаются в виде пар (антецедент, консеквент).
Причём антецедент должен быть представлен в виде конъюнктов элементарных высказываний.
Элементарные высказывания задаются тройкой (переменная, оператор сравнения, переменная или константа),
вместе с правилом будет храниться автоматический список используемых переменных для ускорения поиска соответствующих правил.
Консеквент будет содержать список пар (переменная, значение).
При истинном значении антецедента в модуль трассировки будет отправляться правило вместе с текущими значениями переменных.
Пользователю будут отображаться вычисленные значения целевых переменных с возможными рекомендациями.

Минимальный пример базы знаний:


% Проблемы с кириллицей
\begin{lstlisting}
{
	"rules": [
	{
		"antecedent": [["_t", ">", 36.6]],
		"consequent": ["$result", "температура повышена"],
		"recommendation": "Снизить температуру"
	},
	{
		"antecedent": [
		["_t", "<", 35.5],
		["_P_low", "<", 120]
		],
		"consequent": ["$state", ["*", "_t", "_P_low"]],
		"recommendation": "Низкие температура и давление"
	},
	{
		"antecedent": []
	}
	],
	"input_variables": [
	{
		"name": "_t",
		"type": "float",
		"description": "Текущая температура тела человека"
	},
	{
		"name": "_P_low",
		"type": "int",
		"description": "Нижне давление"
	}
	]
}
\end{lstlisting}

Для упрощения парсинга вводятся следующие соглашения об именовании переменных:

\begin{itemize}
	\item \texttt{\$x} --- вспомогательная переменная;
	\item \texttt{\_x} --- переменная, запрашиваемая у пользователя;
	\item \texttt{@x} --- переменная, запрашиваемая у пользователя, если требуются дополнительные данные (за последний проход не применилось ни одно правило и итоговый результат неизвестен).
\end{itemize}

\subsection{База данных}

Для хранения базы данных будет использоваться словарь Python, хранящий пары (имя переменной, значение).

Так как используется только один экземпляр базы данных на всю экспертную систему, используется паттерн <<Одиночка>> \cite{gammaDesignPatternsElements1995}.

\subsection{Модуль принятия решения}

Экземпляр модуля принятия решения создаётся в ядре.
Затем анализируется база знаний и отправляется запрос в модуль ввода-вывода для получения начальных данных.

После заполнения

\subsection{Модуль ввода-вывода}

Модуль ввода-вывода представлен абстрактным классом со следующими методами: чтение переменной.

\subsubsection{Интерфейс REST API}

Интерфейс пользователя представляется в виде REST API, а для конечного пользователя созданы 2 web-интерфейса на React.

\subsection{Ядро}

Ядро системы представляет собой класс для связи других элементов системы.

При инициализации ядро принимает путь к файлу базы знаний и загружает её в память.
Затем подаёся сигнал модулю ввода-вывода о загрузке начальных данных.
