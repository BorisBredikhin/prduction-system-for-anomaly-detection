\section{Архитектура системы}

Предполагается следующая архитектура системы (рис. \ref{fig:ais}).

\addimghere{img/system_architecture}{1.0}{Архитектура интеллектуальной системы}{fig:ais}

Для построения интерфейсов пользователя будет использоваться tkinter, база данных будет представлена в виде словаря Python, а база знаний будут храниться в SQLite.

\subsection{База знаний}

Правила задаются в виде пар (антецедент, консеквент).
Причём антецедент должен быть представлен в виде конъюнктов элементарных высказываний.
Элементарные высказывания задаются тройкой (переменная, оператор сравнения, переменная или константа),
вместе с правилом будет храниться автоматический список используемых переменных для ускорения поиска соответствующих правил.
Консеквент будет содержать список пар (переменная, значение).
При истинном значении антецедента в модуль трассировки будет отправляться правило вместе с текущими значениями переменных.
Пользователю будут отображаться вычисленные значения целевых переменных с возможными рекомендациями.

\subsection{База данных}

Для хранения базы данных будет использоваться словарь Python, хранящий пары (имя переменной, значение).

\subsection{Модуль принятия решения}
