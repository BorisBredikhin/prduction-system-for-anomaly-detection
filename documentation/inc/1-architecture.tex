\section{Архитектура системы}

Предполагается следующая архитектура системы (рис. \ref{fig:ais}).

\addimghere{img/system_architecture}{1.0}{Архитектура интеллектуальной системы}{fig:ais}

Для построения интерфейсов пользователя будет использоваться tkinter, база данных будет представлена в виде словаря Python, а база знаний будет храниться в JSON.

Работа системы основана на очереди сообщений.

\subsection{База знаний и интерфейс эксперта}

Правила задаются в виде пар (антецедент, консеквент).

Причём антецедент должен быть представлен в виде конъюнктов элементарных высказываний.
Элементарные высказывания задаются тройкой (переменная, оператор сравнения, переменная или константа),
вместе с правилом будет храниться автоматический список используемых переменных для ускорения поиска соответствующих правил.

Консеквент будет содержать список пар (переменная, значение).
При истинном значении антецедента в модуль трассировки будет отправляться правило вместе с текущими значениями переменных.
Пользователю будут отображаться вычисленные значения целевых переменных с возможными рекомендациями.
Также в консеквенте может быть введена новая переменная на основе имеющихся в базе данных значений (выражения записываются в lisp-подобной нотации).

Для облегчения парсинга имена переменной должны начинаться со знака \$

Минимальный пример базы знаний:


% Проблемы с кириллицей
\begin{lstlisting}
{
	"rules": [
	{
		"antecedent": [
		["$t", ">", 36.6]
		],
		"consequent": [
		"result",
		"температура повышена"
		],
		"recommendation": "Снизить температуру"
	},
	{
		"antecedent": [
		["$t", "<", 35.5],
		["$P_low", "<", 120]
		],
		"consequent": [
		"$state",
		{
			"type": "str",
			"description": "Состояние"
		}
		],
		"recommendation": "Низкие температура и давление"
	},
	{
		"antecedent": [
		["$t", "<", 34.5],
		["$P_low", "<", 120]
		],
		"consequent": [
		"$state",
		["+", "$t", ["*", 2, "$P_low"]]
		],
		"recommendation": "Низкие температура и давление"
	}
	],
	"input_variables": [
	{
		"name": "$t",
		"type": "float",
		"description": "Текущая температура тела человека"
	},
	{
		"name": "$P_low",
		"type": "int",
		"description": "Нижне давление"
	}
	]
}
\end{lstlisting}


\subsection{База данных}

Для хранения базы данных используется словарь Python, хранящий пары (имя переменной, значение).

Так как используется только один экземпляр базы данных на всю экспертную систему, используется паттерн <<Одиночка>> \cite{gammaDesignPatternsElements1995}.

\subsection{Модуль принятия решения}

Экземпляр модуля принятия решения создаётся в ядре.
Затем анализируется база знаний и отправляется запрос в модуль ввода-вывода для получения начальных данных.

После заполнения базы знаний модуль в бесконечном цикле итерирует по правилам.
Если на текущей итерации не  выполнилось ни одно правило, то работа алгоритма останавливается и отправляется сообщение об отсутствии подходящих правил.

Если значение антецедента истинно вычисляется консеквент с помощью обхода его дерева в глубину.
В модуль трассировки отправляется  сработавшее правило и значения всех используемых в нём переменных, при наличии добавляется экспертный комментарий.

\subsection{Модуль ввода-вывода}

Модуль ввода-вывода представлен абстрактным классом со следующими методами: чтение переменной, вывод сообщения.

\subsubsection{Интерфейс пользователя}

Разработан web клиент в виде чата, который общается с системой по протоколу websocket

\subsection{Ядро}

Ядро системы представляет собой класс для связи других элементов системы.

При инициализации ядро принимает путь к файлу базы знаний и загружает её в память.
Затем подаёся сигнал модулю ввода-вывода о загрузке начальных данных.
А после загрузки начальных данных сигнал о начале логического вывода.